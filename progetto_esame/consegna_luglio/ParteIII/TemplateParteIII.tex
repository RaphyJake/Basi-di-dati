\documentclass[a4paper]{article}
\usepackage{graphicx} % Required for inserting images
\usepackage{todonotes}
\usepackage{geometry}
\geometry{a4paper, top=3cm, bottom=3cm, left=3.5cm, right=3.5cm} % heightrounded, bindingoffset=5mm}

\title{Progetto Basi di Dati 2024-25 \\
“FANTASANREMO” \\
Parte III}
\author{[Numero Gruppo]
\\
\and [Nome, Cognome e Matricola dei componenti]}
\date{}

\begin{document}

\maketitle


\section{Progettazione Fisica\\}

\subsection{CARICO DI LAVORO\\}

\subsubsection{Q1 - QUERY CON SINGOLA SELEZIONE E NESSUN JOIN\\}

\paragraph*{LINGUAGGIO NATURALE \\} 


\todo[inline]{Riportare in questa sezione l’interrogazione del carico di lavoro con singola selezione e nessun join, in linguaggio naturale}

\paragraph*{SQL \\}

\todo[inline]{Riportare in questa sezione l’interrogazione del carico di lavoro con singola selezione e nessun join, in SQL}



\subsubsection{Q2 - QUERY CON CONDIZIONE DI SELEZIONE COMPLESSA E NESSUN JOIN\\}

\paragraph*{LINGUAGGIO NATURALE\\}

\todo[inline]{Riportare in questa sezione l’interrogazione del carico di lavoro con condizione di selezione complessa e nessun join, in linguaggio naturale}

\paragraph*{SQL \\}

\todo[inline]{Riportare in questa sezione l’interrogazione del carico di lavoro con condizione di selezione complessae nessun join, in SQL}


\subsubsection{Q3 - QUERY CON ALMENO UN JOIN E ALMENO UNA CONDIZIONE DI SELEZIONE\\ }

\paragraph*{LINGUAGGIO NATURALE\\}

\todo[inline]{Riportare in questa sezione l’interrogazione del carico di lavoro  con almeno un join e almeno una condizione di selezione, in linguaggio naturale}

\paragraph*{SQL \\}

\todo[inline]{Riportare in questa sezione l’interrogazione del carico di lavoro con almeno un join e almeno una condizione di selezione, in SQL}


\subsection{1D-PROGETTO FISICO\\}

\todo[inline]{Riportare nella seguente tabella l’elenco degli indici che si intendono creare per: (1) ciascuna query del carico di lavoro individualmente; (2) l’insieme delle query del carico di lavoro, motivando opportunamente, in modo sintetico, le scelte effettuate}


\begin{center}
\begin{footnotesize}
\begin{tabular}{|c|c|p{2cm}|p{2cm}|p{5cm}|}
\hline
{\bf Id query} & {\bf Relazione} & \parbox{2cm}{\bf Chiave di ricerca} & \parbox{2cm}{\bf Tipo (ordinato/hash, clusterizzato/non clusterizzato)} & \parbox{5cm}{\bf Motivazione} \\
\hline
& & & & \\
\hline
\end{tabular}
\end{footnotesize}
\end{center}



\begin{center}
\begin{footnotesize}
\begin{tabular}{|p{6.5cm}|p{6.5cm}|}
\hline
\parbox{5cm}{\bf Schema fisico complessivo per il carico di lavoro} &  \parbox{5cm}{\bf Motivazione} \\
\hline
&  \\
\hline
\end{tabular}
\end{footnotesize}
\end{center}

\subsection{1G-ANALISI PIANI DI ESECUZIONE SCELTI DAL SISTEMA\\}


\subsubsection{Q1 - QUERY CON SINGOLA SELEZIONE E NESSUN JOIN\\}

\paragraph*{PIANO DI ESECUZIONE SCELTO DAL SISTEMA PRIMA DELLA CREAZIONE DELLO SCHEMA FISICO\\}

\todo[inline]{Riportare in questa sezione il piano di esecuzione scelto da PostgreSQL prima della creazione dello schema fi-sico complessivo, in formato testuale}

\paragraph*{PIANO DI ESECUZIONE SCELTO DAL SISTEMA DOPO DELLA CREAZIONE DELLO SCHEMA FISICO\\}

\todo[inline]{Riportare in questa sezione il piano di esecuzione scelto da PostgreSQL dopo della creazione dello schema fisi-cocomplessivo, in formato testuale}

\paragraph*{CONFRONTO TRA I DUE PIANI\\}

\todo[inline]{Riportare nella seguente tabella i tempi di esecuzione per i piani ottenuti prima e dopo la creazione dello schema fisico complessivo, giustificando i piani e i tempi ottenuti}


\begin{center}
\begin{footnotesize}
\begin{tabular}{|p{3cm}|p{3cm}|p{7cm}|}
\hline
\parbox{3cm}{\bf Tempo esecuzione PRIMA} & \parbox{3cm}{\bf Tempo esecuzione DOPO} &  \parbox{7cm}{\bf Motivazione} \\
\hline
& &  \\
\hline
\end{tabular}
\end{footnotesize}
\end{center}

\subsubsection{Q2 - QUERY CON CONDIZIONE DI SELEZIONE COMPLESSA E NESSUN JOIN\\}

\paragraph*{PIANO DI ESECUZIONE SCELTO DAL SISTEMA PRIMA DELLA CREAZIONE DELLO SCHEMA FISICO\\}

\todo[inline]{Riportare in questa sezione il piano di esecuzione scelto da PostgreSQL prima della creazione dello schema fi-sico complessivo, in formato testuale}

\paragraph*{PIANO DI ESECUZIONE SCELTO DAL SISTEMA DOPO DELLA CREAZIONE DELLO SCHEMA FISICO\\}

\todo[inline]{Riportare in questa sezione il piano di esecuzione scelto da PostgreSQL dopo della creazione dello schema fisi-co complessivo, in formato testuale}

\paragraph*{CONFRONTO TRA I DUE PIANI\\}

\todo[inline]{Riportare nella seguente tabella i tempi di esecuzione per i piani ottenuti prima e dopo la creazione dello schema fisico complessivo, giustificando i piani e i tempi ottenuti}


\begin{center}
\begin{footnotesize}
\begin{tabular}{|p{3cm}|p{3cm}|p{7cm}|}
\hline
\parbox{3cm}{\bf Tempo esecuzione PRIMA} & \parbox{3cm}{\bf Tempo esecuzione DOPO} &  \parbox{7cm}{\bf Motivazione} \\
\hline
& &  \\
\hline
\end{tabular}
\end{footnotesize}
\end{center}



\subsubsection{Q3 - QUERY CON ALMENO UN JOIN E ALMENO UNA CONDIZIONE DI SELEZIONE \\}

\paragraph*{PIANO DI ESECUZIONE SCELTO DAL SISTEMA PRIMA DELLA CREAZIONE DELLO SCHEMA FISICO\\}

\todo[inline]{Riportare in questa sezione il piano di esecuzione scelto da PostgreSQL prima della creazione dello schema fi-sico complessivo, in formato testuale}

\paragraph*{PIANO DI ESECUZIONE SCELTO DAL SISTEMA DOPO DELLA CREAZIONE DELLO SCHEMA FISICO\\}

\todo[inline]{Riportare in questa sezione il piano di esecuzione scelto da PostgreSQL dopo della creazione dello schema fisi-co complessivo, in formato testuale}

\paragraph*{CONFRONTO TRA I DUE PIANI\\}

\todo[inline]{Riportare nella seguente tabella i tempi di esecuzione per i piani ottenuti prima e dopo la creazione dello schema fisico complessivo, giustificando i piani e i tempi ottenuti}


\begin{center}
\begin{footnotesize}
\begin{tabular}{|p{3cm}|p{3cm}|p{7cm}|}
\hline
\parbox{3cm}{\bf Tempo esecuzione PRIMA} & \parbox{3cm}{\bf Tempo esecuzione DOPO} &  \parbox{7cm}{\bf Motivazione} \\
\hline
& &  \\
\hline
\end{tabular}
\end{footnotesize}
\end{center}




\section{CONTROLLO DELL’ACCESSO\\}

\subsection{GERARCHIA TRA I RUOLI \\}

\subsubsection{GERARCHIA\\}

\todo[inline]{Riportare in questa sezione la gerarchia che si intende definire tra i quattro ruoli}

\subsubsection{MOTIVAZIONE GERARCHIA}

\todo[inline]{Riportare in questa sezione una motivazione per la gerarchia proposta}


\subsection{ASSEGNAZIONE PRIVILEGI SPECIFICI AI RUOLI}

\todo[inline]{Riportare nella prima colonna della seguente tabella le relazioni considerate e in ciascuna altra cella (i,j) i privilegi specifici (quindi non acquisiti tramite la gerarchia) che si intendono assegnare al ruolo j sulla tabella }



\begin{center}
\begin{footnotesize}
\begin{tabular}{|c|p{2.7cm}|p{2.7cm}|p{2.7cm}|p{2.7cm}|}
\hline
{\bf Relazione} & \parbox{2.7cm}{\bf Amministratore del FantaSarremo} & \parbox{2.7cm}{\bf Utente} &  \parbox{2.7cm}{\bf Amministratore lega} & \parbox{2.7cm}{\bf Proprietario lega} \\
\hline
& &  & & \\
\hline
\end{tabular}
\end{footnotesize}
\end{center}


\end{document}
