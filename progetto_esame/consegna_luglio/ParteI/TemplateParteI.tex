\documentclass[a4paper,12pt]{article}
\usepackage[utf8]{inputenc}
\usepackage{geometry}
\geometry{top=2cm, bottom=2cm, left=2.5cm, right=2.5cm}
\usepackage{graphicx}
\usepackage{titlesec}
\usepackage{enumitem}
\usepackage{fancyhdr}
\usepackage{setspace}
\usepackage{todonotes}
\pagestyle{fancy}
\fancyhf{}
\rhead{Progetto Basi di Dati 2024-25}
\lhead{“FANTASANREMO”}
\cfoot{\thepage}

\titleformat{\section}{\normalfont\Large\bfseries}{\thesection.}{1em}{}
\titleformat{\subsection}{\normalfont\large\bfseries}{\thesubsection}{1em}{}

\begin{document}

\begin{center}
    {\LARGE \textbf{Progetto Basi di Dati 2024-25}}\\[.5cm]
    {\Large \textbf{“FANTASANREMO”}}\\[.5cm]
    {\Large \textbf{PARTE I}}\\[.8cm]
    [Numero Gruppo]\\[.3cm]
    [Nome, Cognome e Matricola dei componenti]
\end{center}

\vspace{1cm}

\section{REQUISITI RISTRUTTURATI}
\noindent
[Riportare in questa sezione la specifica dei requisiti ristrutturata in modo da eliminare ambiguità, evidenziando eventuali modifiche effettuate rispetto alla specifica fornita]

\section{PROGETTO CONCETTUALE}
\subsection{Schema Entity Relationship}
[Riportare in questa sezione il diagramma ER, indicando anche il tipo delle gerarchie di generalizzazione]

\subsection{Dizionario Dati – Domini degli Attributi}
[Specificare i domini degli attributi non ovvi]

\subsection{Vincoli Non Esprimibili nel Diagramma}
[Specificare i vincoli NON esprimibili nel diagramma ER, inclusi quelli relativi a attributi e associazioni ridondanti]

\begin{table}[h!]
    \centering
    \begin{tabular}{|c|c|p{8cm}|}
        \hline
        \textbf{Numero Vincolo} & \textbf{Entità e/o Associazioni Coinvolte} & \textbf{Vincolo in Linguaggio Naturale} \\ \hline
        & & \\ \hline
        & & \\ \hline
    \end{tabular}
    \caption{Vincoli Non Esprimibili nel Diagramma ER}
\end{table}

\subsection{Commenti e Scelte Effettuate}
[se necessario commentare qui le scelte effettuate]

\newpage
\section{PROGETTO LOGICO}
\subsection{Schema ER Ristrutturato}
[inserire qui il diagramma ER ristrutturato]

\subsection{Domini degli Attributi}
[eventuali modifiche dei domini degli attributi e informazioni sui domini di eventuali attributi introdotti]

\subsection{Vincoli}
[modifiche all'elenco di vincoli del modello concettuale (nuovi vincoli, eventuali vincoli eliminati o modificati)]

\subsection{Ristrutturazione Gerarchie}
[discussione delle scelte fatte per eliminare le gerarchie di generalizzazione]

\subsection{Schema Logico}
[schema logico relazionale nella notazione vista a lezione]

\subsection{Verifica di Qualità dello Schema}
[verifica delle forme normali ed eventuali ottimizzazioni applicate tenendo in considerazione il carico di lavoro]

\subsection{Schema SQL in Forma Grafica}
[diagramma che visualizza lo script SQL di creazione dello schema in forma grafica ottenuto con DataGrip (vedi Aulabweb per come crearla dallo script)]

\subsection{Uso di Intelligenza Artificiale Generativa}
[Specificare se e come è stata utilizzata l’intelligenza artificiale generativa per realizzare le varie parti del progetto. Indicare anche se sono stati effettuati esperimenti di uso di AI generativa relativi al progetto che non hanno prodotto alcun contributo effettivamente utilizzato nel progetto consegnato, commentandone i risultati]

\end{document}
